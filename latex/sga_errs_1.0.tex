\documentclass[11pt,a4paper]{article}
\usepackage{booktabs}
\usepackage{placeins}
\usepackage{bm}
\usepackage{array}

\raggedright

\usepackage[T1]{fontenc}
\usepackage[utf8]{inputenc}
\usepackage{lmodern}
\usepackage[margin=30mm]{geometry}
\setlength{\parindent}{0pt}
\setlength{\parskip}{0.75em}

\usepackage{amsmath,amssymb}
\usepackage[hidelinks]{hyperref}

\title{\textbf{Evidence Ratio Reporting Standard}}
\author{Swiss Genomics Association}
\date{2026}

\begin{document}
\pagenumbering{gobble}

\maketitle

\noindent
\textbf{Version 1.0}\\
\noindent\textbf{Standard identifier}: SGA-ERRS-1.0\\
\noindent This document defines a normative reporting standard published by the Swiss Genomics Association.

\section{Introduction}

Statistical results are increasingly stored, compared, and reused in databases and analytic platforms rather than read only through journal articles. In these settings, results from different analyses must be reported in a consistent and self contained way to support comparison and reproducibility.

The Evidence Ratio Reporting Standard (ERRS) defines a reporting completeness standard for statistical evidence derived from likelihood based analyses. Its purpose is to specify the minimal set of quantities and metadata that \textbf{MUST} be reported whenever an evidence ratio is reported.

This standard applies only once an evidence ratio has been chosen as a reporting quantity and does not require, recommend, or encourage the use of evidence ratios in any analysis. ERRS standardises reporting requirements only. It does not define analytical methods, modelling choices, inferential procedures, or decision rules.

The ERRS is designed to support structured, interoperable representation of statistical evidence across heterogeneous analyses, suitable for use in databases, registries, and analytic systems linked to electronic health record infrastructures.


\section{Scope}

This standard governs reporting requirements only and applies exclusively to analyses that already report an evidence ratio.

\subsection{In scope}

The ERRS defines:

\begin{itemize}
\item the normative definition of an evidence ratio report
\item the required components that \textbf{MUST} accompany any reported evidence ratio
\item minimal coherence constraints between reported components
\item minimal integrity and provenance requirements
\end{itemize}

\subsection{Out of scope}

The ERRS does not define:

\begin{itemize}
\item how statistical models are constructed or selected
\item how null or alternative models are chosen
\item how reported quantities should be interpreted
\item thresholds, classifications, or decision criteria
\item clinical, scientific, or regulatory conclusions
\end{itemize}

\section{Terminology and conventions}

The key words \textbf{MUST}, \textbf{MUST NOT}, \textbf{SHOULD}, and \textbf{MAY} are to be interpreted as described in RFC 2119.

An \textbf{analysis} is a statistical procedure applied to observed data that yields one or more model based quantities.

A \textbf{model} is a statistical specification defining how observed data are generated.

A \textbf{null model} represents a specified reference model, typically encoding absence of an effect or association.

An \textbf{alternative model} represents a specified model distinct from the null model.

An \textbf{effect estimate} is a scalar quantity summarising a parameter or function of parameters of interest under a specified model.

An \textbf{uncertainty interval} is an interval estimate associated with an effect estimate, reflecting uncertainty under a specified inferential framework.

An \textbf{evidence ratio} is a ratio of likelihoods or marginal likelihoods comparing two specified models for the observed data.

\section{Definition of an evidence ratio report}

An \textbf{evidence ratio report} is the normative reporting object defined by this standard.

An evidence ratio report is a structured report instance that contains a set of quantities derived from a single analysis and dataset, including an evidence ratio and its required accompanying information.

This standard does not define the interpretation or use of an evidence ratio report. It defines only the information that \textbf{MUST} be present for reporting to be compliant.

\section{Required components}

Each evidence ratio report \textbf{MUST} include the following components, all derived from the same statistical model and dataset:

\begin{itemize}
\item an effect estimate
\item an uncertainty interval corresponding to that estimate
\item a likelihood based evidence ratio, reported on the log$_{10}$ scale
\item an explicit definition of the null model
\item an explicit definition of the alternative model
\end{itemize}

No ordering or priority is implied by this list.
A report that omits any of these components is not compliant with this standard.

\section{Component definitions}

\subsection{Effect estimate}

The effect estimate \textbf{MUST} be a scalar quantity derived from the specified model.
The method used to obtain the effect estimate \textbf{MUST} be stated or be unambiguous from context.
The scale and parameterisation of the effect estimate \textbf{MUST} be stated or be unambiguous from context.

\subsection{Uncertainty interval}

The uncertainty interval \textbf{MUST} correspond to the reported effect estimate.
The interval level or credibility definition \textbf{MUST} be stated.
The method used to construct the uncertainty interval \textbf{MUST} be stated.
The uncertainty interval \textbf{MUST} be derived from the same statistical model and data as the effect estimate.
Examples include, but are not limited to, confidence intervals, credible intervals, and likelihood based intervals.

\subsection{Evidence ratio}

Let \( x \) denote the observed data.
Let \( m_0(x) \) denote the likelihood or marginal likelihood of the data under the null model.
Let \( m_1(x) \) denote the likelihood or marginal likelihood of the data under the alternative model.

The evidence ratio is defined as:
\[
E(x) = \frac{m_1(x)}{m_0(x)}.
\]

The evidence ratio \textbf{MUST} be reported on a logarithmic scale.
By default, the log$_{10}$ scale \textbf{MUST} be used and \textbf{SHOULD} be preferred for tabular and comparative reporting.
The raw evidence ratio \textbf{MAY} be reported in addition to the logarithmic form.
If an alternative logarithmic base or the raw evidence ratio is reported, the scale and transformation \textbf{MUST} be stated explicitly.
The evidence ratio \textbf{MUST} be derived from the same data and model specifications as the reported effect estimate and uncertainty interval.

\section{Coherence requirements}

All components of an evidence ratio report \textbf{MUST} be derived from internally consistent modelling assumptions.
The effect estimate, uncertainty interval, and evidence ratio \textbf{MUST} correspond to the same model specification and dataset.
If likelihoods are maximised over parameters, this \textbf{MUST} be stated.
If likelihoods are marginalised over parameters, this \textbf{MUST} be stated.

\section{Integrity and provenance}

An ERRS compliant report \textbf{MUST} include or reference sufficient information to allow independent reproduction of the reported quantities.
At a minimum, the report \textbf{MUST} specify:

\begin{itemize}
\item the statistical model used
\item the null model definition
\item the alternative model definition
\item whether likelihoods are maximised or marginalised
% \item the ERRS version applied
\end{itemize}

\section{Non goals}

This standard does not define how evidence ratio reports should be interpreted, combined, or used for inference or decision making.
The standard does not assign meaning, strength, or sufficiency to any reported quantity.

\section{Examples (informative)}

This section provides non normative examples illustrating how evidence ratio reports may be presented in practice.
The examples demonstrate compliance with this standard across distinct and widely used classes of biomedical statistical analysis.
No interpretation, thresholding, or inferential guidance is implied.

\subsection{Single analysis report}

Consider a randomised drug study evaluating the effect of a treatment on a continuous clinical outcome, analysed using linear regression.
The analysis estimates the treatment effect, reports its uncertainty, and compares a model with a non zero treatment effect to a null model of no effect.

\textbf{Table~\ref{tab:single-example}} shows the corresponding evidence ratio report.
All components required by this standard are reported explicitly for a single analysis.

\begin{table}[h]
\centering
\caption{\textit{Example of a single evidence ratio report.}}
\label{tab:single-example}
\begin{tabular}{ll}
\toprule
\textbf{Component} & \textbf{Reported value} \\
\midrule
Effect estimate &
$-0.59$ \\
Uncertainty interval (95\%) &
$[-1.26,\;0.08]$ \\
Evidence ratio &
$\log_{10} E(x) = 0.17$ \\
Null model &
No treatment effect \\
Alternative model &
Non zero treatment effect \\
Model specification &
Linear regression with Gaussian errors \\
\bottomrule
\end{tabular}
\end{table}

\subsection{Multiple analyses reported using a common schema}

To illustrate how the same reporting structure applies across heterogeneous analyses, consider the following set of studies:

\begin{itemize}
\item \textbf{Randomised drug trial assessing a treatment effect on a continuous outcome.}
Null model: no treatment effect.
Alternative model: non zero treatment effect.
Model specification: Gaussian linear regression.

\item \textbf{Before after study testing whether a biomarker mean differs from zero.}
Null model: mean equals zero.
Alternative model: mean differs from zero.
Model specification: normal mean model.

\item \textbf{Two arm clinical comparison evaluating differences between treatment groups.}
Null model: equal group means.
Alternative model: unequal group means.
Model specification: two sample normal model.

\item \textbf{Case control study analysing disease status using a $2 \times 2$ contingency table.}
Null model: odds ratio equals one.
Alternative model: odds ratio differs from one.
Model specification: $2 \times 2$ likelihood.

\item \textbf{Observational study assessing association between a continuous exposure and outcome.}
Null model: regression coefficient equals zero.
Alternative model: regression coefficient differs from zero.
Model specification: linear regression.

\item \textbf{Time to event study analysing treatment effects on survival time.}
Null model: no acceleration effect.
Alternative model: acceleration differs from zero.
Model specification: accelerated failure time model.

\item \textbf{Survival study assessing differences in hazard between treatment groups.}
Null model: hazard ratio equals one.
Alternative model: hazard ratio differs from one.
Model specification: Cox proportional hazards model.
\end{itemize}

Each study uses a different statistical model and estimand, but all produce an evidence ratio report compliant with this standard.
The first row in \textbf{Table~\ref{tab:multi-example}} reproduces the analysis shown in \textbf{Table~\ref{tab:single-example}}, demonstrating continuity between single and multiple report settings.

\begin{table}[h]
\centering
\caption{\textit{Example of multiple evidence ratio reports across heterogeneous analyses.}}
\label{tab:multi-example}
%\begin{tabular}{p{7.0cm} p{1.6cm} p{3.2cm} p{2.2cm}}
\begin{tabular}{
p{6.5cm}
>{\raggedleft\arraybackslash}p{1.4cm}
>{\raggedleft\arraybackslash}p{2.8cm}
>{\raggedleft\arraybackslash}p{1.4cm}
} % WARNING without array package this will case a crash with xelatex
\toprule
\textbf{Analysis type} &
\textbf{Effect}\newline \textbf{estimate} &
\textbf{Uncertainty}\newline \textbf{interval (95\%)} &
$\boldsymbol{\log_{10} E(x)}$ \\
\midrule
Linear regression \newline \textit{treatment effect on continuous outcome}
& $-0.59$ & $[-1.26,\;0.08]$ & $0.17$ \\
\addlinespace
One sample mean test \newline \textit{mean difference from zero}
& $0.39$ & $[0.18,\;0.59]$ & $3.05$ \\
\addlinespace
Two sample mean test \newline \textit{difference between treatment groups}
& $-1.32$ & $[-1.77,\;-0.88]$ & $14.42$ \\
\addlinespace
Binary outcome association \newline \textit{$2 \times 2$ contingency table}
& $0.48$ & $[-0.13,\;1.09]$ & $0.52$ \\
\addlinespace
Regression coefficient \newline \textit{continuous exposure}
& $-5.73$ & $[-6.59,\;-4.88]$ & $18.61$ \\
\addlinespace
Time to event analysis \newline \textit{accelerated failure time model}
& $-0.27$ & $[-0.57,\;0.03]$ & $0.65$ \\
\addlinespace
Survival analysis \newline \textit{Cox proportional hazards model}
& $0.96$ & $[0.64,\;1.28]$ & $7.50$ \\
\bottomrule
\end{tabular}
\end{table}


Each row in \textbf{Table~\ref{tab:multi-example}} corresponds to a distinct and widely used statistical analysis.
Although the underlying models, estimands, and effect scales differ, each analysis is reported using the same evidence ratio reporting structure defined by this standard.
This illustrates how heterogeneous biomedical evidence can be expressed on a common likelihood based evidential scale without altering the underlying statistical methods.

\FloatBarrier
\section{Versioning}

This document defines \textbf{ERRS version 1.0}.  
Future revisions \textbf{MUST} preserve the definitions and reporting semantics specified herein.

\end{document}
